\documentclass[11pt]{article}

\usepackage{hyperref}
\usepackage{geometry}
\geometry{margin=0.6in}

\title{\textbf{ATAC-Seq technical paper}}
\author{Irina Pulyakhina}
\date{}
\begin{document}

\maketitle

\renewcommand\labelitemi{\tiny$\bullet$}

\section{Introduction}

DONE


\section{Materials and Methods}
\subsection{Simulated data}
\begin{enumerate}
\itemsep0em
  \item Use the CD14 annotation.
  \item Select regions annotated as regulatory elements.
  \item Create reads only over selected regulatory elements.
  \item Coverage of promoters, enhancers and TSS should be significantly
        higher than the coverage of other regions.
  \item Use \href{http://www.ebi.ac.uk/goldman-srv/simNGS/}{simNGS} or
        \href{http://samtools.sourceforge.net/}{wgsim} to simulate
        paired-end Illumina HiSeq reads.
  \item ``Titration'': simulate reads in different quantities to
        identify the minimal required coverage for reliable analysis
        results.
\end{enumerate}

\subsection{Biological data}

\subsection{General bioinformatic analysis} Overview of the pipeline:
\begin{enumerate}
\itemsep0em 
  \item Quality control.
  \item Adapter trimming.
  \item Alignment.
  \item Filtering the alignment results to select the best quality data. 
  \item Peak calling.
  \item Annotation.
\end{enumerate}

\subsection{Comparing different peak callers}
\begin{enumerate}
\itemsep0em
  \item Three peak callers:
        \href{https://pypi.python.org/pypi/MACS2}{macs2}, 
        \href{http://fureylab.web.unc.edu/software/fseq/}{F-Seq} and 
        \href{http://www.uwencode.org/proj/hotspot-ptih/}{HotSpot}
  \item How we compare the results:
  \begin{itemize}
    \item Number of called peaks.
    \item Average peak width.
    \item Average peak height.
    \item Number of narrow peaks (below 200bp) (\textbf{NP}).
    \item Number of wide peaks (above 200bp) (\textbf{WP}).
    \item Ratio \textbf{NP}/\textbf{WP}.
    \item Signal-to-noise ratio (reads mapped to peaks / reads mapped
          outside peaks).
    \item Enrichment of different types of regulatory elements (heatmap).
  \end{itemize}
\end{enumerate}

\section{Results}

\subsection{Pinechrom}
\begin{enumerate}
\itemsep0em
  \item New, first publicly available open-source pipeline to analyze ATAC data
  \item Flexible execution, consists of modules and allows different types of analysis
  \item Module 1 -- \textit{pinechrom\_general}: basic analysis performed per sample, 
         mapping, filtering reads, assessing quality of mapped reads; peak calling,
         assessing quality of peaks.
  \item Module 2 -- \textit{pinechrom\_diff}: comparison of samples from different
         conditions: getting consensus peak calls from multiple replicates (if available), 
         comparing consensus peaks, reporting peaks specific per condition, common 
         between two conditions and differentially binding, common between two
         conditions and binding with the same efficiency.
  \item Module 3 -- integrating external chromatin data (DNase, FAIRE, methylation,
         histon marks, annotation of regulatory elements).
  \item Module 4 -- integrating GWAS, eQTL or RNA-Seq (any expression) data.
\end{enumerate}

\subsection{Comparing peak callers}
\begin{enumerate}
\itemsep0em
  \item Overview of the results generated by each peak caller on the 
         simulated data.
  \item Comparing results on simulated samples with different number of 
         reads.
  \item Comparing results of three peak callers generated on simulated 
         data.
  \item Calling peaks on DNase with new peak callers and comparing the results.
\end{enumerate}

\subsection{Comparing conditions}
\begin{enumerate}
\itemsep0em
  \item General quality of different conditions -- statistics (number of peaks,
         signal-to-noise ratio, number of peaks present in three replicates, overlap 
         with DNase peaks, overlap with K562 annotation, etc).
  \item Specifics / peculiarities of each condition (fresh / frozen / fixed 1 day /
         fixed 3 days / fixed 5 days).
  \item 
\end{enumerate}

\subsection{Downsampling}
\begin{enumerate}
\itemsep0em
  \item To identify the minima required number of sequenced reads (depth of
         sequencing) to get reliable results.
  \item Can we see the saturation point?
  \item Which peaks are we loosing (DNase, any specific annotation)
\end{enumerate}

\subsection{Analyzing immunological data}
\begin{enumerate}
\itemsep0em
  \item Effect of cryopreservation / fixation on different cell types.
  \item General statistical comparison with K562 data (number of peaks,
         signal-to-noise ratio, overlap with DNase, overlap with know regulatory
         elements).
\end{enumerate}

\section{Discussion}
\begin{itemize}
\itemsep0em
  \item Performance of different peak callers.
  \item Which peak caller is more suitable for which biological question.
  \item Minimal number of reads recommended for getting reliable/
        reasonable analysis results.
  \item Reproducibility of ATAC-Seq in different material (fresh/frozen/
        fixed).
  \item Reproducibility of ATAC-Seq of different ``age'' (1, 3, 7 days).
  \item Which type of biological material is recommended to use for
        getting the most reliable/reproducible results.	
\end{itemize}

\end{document}







