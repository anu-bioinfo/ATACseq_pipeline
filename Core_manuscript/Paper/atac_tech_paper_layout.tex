\documentclass[11pt]{article}

\usepackage{hyperref}

\title{\textbf{ATAC-Seq technical paper}}
\author{Irina Pulyakhina}
\date{}
\begin{document}

\maketitle

\renewcommand\labelitemi{\tiny$\bullet$}

\section{Introduction}
\begin{enumerate}
\itemsep0em
  \item Chromatin structure and its biological role/importance.
  \item Chromatin accessibility assays (CAA).
  \item Regulatory elements and where they are located more often (open/close chromatin).
  \item Analysis of CAA -- calling peaks to identify open/closed regions of DNA.
  \item Current CAA and their disadvantages.
  \item Introduction of ATAC-Seq.
  \item Advantages of ATAC-Seq.
  \item Lack of expertise in ATAC-Seq.
  \item Requirement in extensive overview of ATAC-Seq lab preparation and bioinformatic analysis.
  \item We will come up with recommendations on how to get the most out of ATAC-Seq data.
\end{enumerate}

\section{Materials and Methods}
\subsection{Simulated data}
\begin{enumerate}
\itemsep0em
  \item Use the CD14 annotation.
  \item Select regions annotated as regulatory elements.
  \item Create reads only over selected regulatory elements.
  \item Coverage of promoters, enhancers and TSS should be significantly
        higher than the coverage of other regions.
  \item Use \href{http://www.ebi.ac.uk/goldman-srv/simNGS/}{simNGS} or
        \href{http://samtools.sourceforge.net/}{wgsim} to simulate
        paired-end Illumina HiSeq reads.
  \item ``Titration'': simulate reads in different quantities to
        identify the minimal required coverage for reliable analysis
        results.
\end{enumerate}

\subsection{Biological data}

\subsection{General bioinformatic analysis} Overview of the pipeline:
\begin{enumerate}
\itemsep0em 
  \item Quality control.
  \item Adapter trimming.
  \item Alignment.
  \item Filtering the alignment results to select the best quality data. 
  \item Peak calling.
  \item Annotation.
\end{enumerate}

\subsection{Comparing different peak callers}
\begin{enumerate}
\itemsep0em
  \item Three peak callers:
        \href{https://pypi.python.org/pypi/MACS2}{macs2}, 
        \href{http://fureylab.web.unc.edu/software/fseq/}{F-Seq} and 
        \href{http://www.uwencode.org/proj/hotspot-ptih/}{HotSpot}
  \item How we compare the results:
  \begin{itemize}
    \item Number of called peaks.
    \item Average peak width.
    \item Average peak height.
    \item Number of narrow peaks (below 200bp) (\textbf{NP}).
    \item Number of wide peaks (above 200bp) (\textbf{WP}).
    \item Ratio \textbf{NP}/\textbf{WP}.
    \item Signal-to-noise ratio (reads mapped to peaks / reads mapped
          outside peaks).
    \item Enrichment of different types of regulatory elements (heatmap).
  \end{itemize}
\end{enumerate}

\section{Results}

\subsection{Comparing peak callers}
\begin{enumerate}
\itemsep0em
  \item Overview of the results generated by each peak caller on the 
        simulated data.
  \item Comparing results on simulated samples with different number of 
        reads.
  \item Comparing results of three peak callers generated on simulated 
        data.
\end{enumerate}

\subsection{Analyzing biological data}

\section{Discussion}
\begin{itemize}
\itemsep0em
  \item Performance of different peak callers.
  \item Which peak caller is more suitable for which biological question.
  \item Minimal number of reads recommended for getting reliable/
        reasonable analysis results.
  \item Reproducibility of ATAC-Seq in different material (fresh/frozen/
        fixed).
  \item Reproducibility of ATAC-Seq of different ``age'' (1, 3, 7 days).
  \item Which type of biological material is recommended to use for
        getting the most reliable/reproducible results.	
\end{itemize}

\end{document}






